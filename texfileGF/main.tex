\documentclass{report}
\usepackage[utf8]{vietnam}
\usepackage[utf8]{inputenc}
\usepackage{anyfontsize,fontsize}
\changefontsize[13pt]{13pt}
\usepackage{commath}
\usepackage{blindtext}
\usepackage{parskip}
\usepackage{xcolor}
\usepackage{amssymb}
\usepackage{slashed,cancel}
\usepackage{indentfirst}
\usepackage{pdfpages}
\usepackage{graphicx}


\usepackage{nccmath}
\usepackage{mathtools}
\usepackage{amsfonts}
\usepackage{amsmath,systeme,bbold}
\usepackage[thinc]{esdiff}
\usepackage{hyperref}
\usepackage{dirtytalk,bm,physics,mathrsfs,enumitem}

\usepackage{lipsum}
\usepackage{fancyhdr}
%footnote
\pagestyle{fancy}
\renewcommand{\headrulewidth}{0pt}%
\fancyhf{}%
\fancyfoot[L]{Vật lý Lý thuyết}%
\fancyfoot[C]{\hspace{4cm} \thepage}%



\usepackage{geometry}
\geometry{
	a4paper,
	total={170mm,257mm},
	left=20mm,
	top=20mm,
}



\title{\Huge{Hàm suy rộng\\Hàm Green}}

\hypersetup{
	colorlinks=true,
	linkcolor=red,
	filecolor=magenta,      
	urlcolor=cyan,
	pdftitle={Hàm Green},
	pdfpagemode=FullScreen,
}

\urlstyle{same}
\newcommand{\s}{\ast}


\newcommand{\image}[1]{
	\begin{center}
		\includegraphics[width=0.5\textwidth]{pic/#1}
	\end{center}
}
\renewcommand{\l}{\ell}
\newcommand{\dps}{\displaystyle}

\newcommand{\f}[2]{\dfrac{#1}{#2}}
\newcommand{\at}[2]{\bigg\rvert_{#1}^{#2} }

\renewcommand{\baselinestretch}{1.5}



\begin{document}
\setlength{\parindent}{20pt}
\newpage
\author{TRẦN KHÔI NGUYÊN \\ VẬT LÝ LÝ THUYẾT}
\maketitle
\chapter*{Preface}
Giáo trình: Applie Functional Analasis(Griffel), Introductory Functional Analasis(Kreyzig), Equations of Mathematical Physics(Vladimirov), Mathmatical Methods (David Skinner). Có thể xem qua thêm như là Boas, Arfken \& Weber.
\tableofcontents
\chapter{Lý thuyết Sturm-Liouville}
\section{Ma trận tự phó(tự liên hợp)(self-adjoint)}
Cho hai không gian vector: $V$ và $W$
\begin{align*}
	V:dim             & = n                   \\
	W:dim             & = m                   \\
	M: V              & \rightarrow W         \\
	\Rightarrow W=M.V & \Leftrightarrow w=M.v \\
\end{align*}
trong đó M: ánh xạ tuyến tính\\
Notation cho tích trong(innner product): (a,b)
\begin{align*}
	\mathbf{M}_{ai} & =(\mathbf{w}_a,M\mathbf{v}_i) \\
	                & =(w_a^T)^{\ast}Mv_i
\end{align*}

Khi $m=n$ có nghĩa là $\mathbf{M}$ là ma trận vuông $nxn$, ta có được:
\begin{align*}
	\lambda_i = det(\mathbf{M}-\lambda\mathbf{I}) = 0
\end{align*}
Một ma trận $\mathbf{M}$ được gọi là ma trận tự phó khi và chỉ khi:
liên hợp Hermit của nó là $\mathbf{M}^\dagger \equiv (\mathbf{M^{T}})^{\ast}= \mathbf{M}$. Ta có thể định nghĩa điều này một cách gọn gàng hơn thông qua notation tích trong: với hai vector ($\mathbf{u,v}$) = $\mathbf{u^\dagger} \cdot \mathbf{v}$, khi ma trận $\mathbf{B}$ là phó của ma trận $\mathbf{A}$ nếu như:
\begin{align}
	(\mathbf{Bu,v}) = (\mathbf{u,Av})
\end{align}
bởi vì $\mathbf{Bu}^\dagger =\mathbf{u}^\dagger \mathbf{B}^\dagger $.
\section{Các tính chất của ma trận tự phó}
\begin{itemize}
	\item Trị riêng của nó phải là thực.
	\item Các vector riêng ứng với 1 trị riêng thì trực giao với tích trong.
	\item Tạo thành cơ sở trực chuẩn.
	\item Nến định thức $\neq$ 0 $\Rightarrow$ tồn tại ma trận nghịch đảo, và các trị riêng đều khác 0.
\end{itemize}
%%%%%%%%%%%%% chứng minh xem trong tập
\section{Toán tử vi phân(Differential operators)}
Toán tử vi phân tuyến tính $\mathcal{L}$, Đây chỉ là tổ hợp tuyến tính của các đạo hàm với các hệ số có thể trở thành hàm của $x$. $\mathcal{L}$ được gọi là toán tử vi phân tuyến tính bậc $p$ khi:
\begin{align*}
	\mathcal{L} = A_p(x)\dfrac{d^p}{dx^p} + A_{p-1}(x)\dfrac{d^{p-1}}{dx^{p-1}} +...+A_1(x)\dfrac{d}{dp} + A_0(x)
\end{align*}
Khi nó đánh lên một hàm $y(x)$(trơn). Ta có $\mathcal{L}(c_1y_1+c_2y_2)=\mathcal{L}(c_1y_1)+\mathcal{L}(c_2y_2) $\\
Ta quan tâm đến phương trình bậc 2 $\mathcal{L}$
\begin{align}
	\mathcal{L}y = \left[ P(x)\dfrac{d^2}{dx^2}+ R(x)\dfrac{d}{dx} - Q(x) \right] y
\end{align}
* Phương trình thuần nhất:
\begin{align*}
	Ly(x) = 0 \rightarrow y = c_1y_1(x) + c_2y_2(x)
\end{align*}
trong đó $c_i$ : complementary function\\
* Phương trình không thuần nhất:
\begin{align*}
	Ly(x)=f(x): y = y_c + y_p
\end{align*}
trong đó: $y_p$ là nghiệm riêng(particular function)\\
* Toán tử vi phân tự phó
\begin{align}
	Ly(x) & = \dfrac{d}{dx}\left(p(x)\dfrac{dy}{dx}\right) - q(x)y\nonumber                       \\
	      & = \left(\dfrac{d}{dx}p(x)\right)\dfrac{dy}{dx} + \dfrac{d^y}{dx^2}p(x)-q(x)y\nonumber \\
	      & = p'(x)\dfrac{dy}{dx} + p(x)\dfrac{d^2y}{dx^2}-q(x)y
\end{align}
Đặt $p'(x) = R(x)$ $\rightarrow$ nghiệm này không tổng quát\\
Giả sử $P(x)\neq0$
\begin{align}
	\dfrac{dy}{dx^2}+\dfrac{R(x)}{P(x)}\dfrac{dy}{dx} - \dfrac{Q(x)}{P(x)}y(x) \\
	\dfrac{dy}{dx^2}+\dfrac{p'(x)}{p(x)}\dfrac{dy}{dx} - \dfrac{q(x)}{p(x)}y(x)
\end{align}
Mối liên hệ giữa p,q và P,Q
\[
	\Rightarrow
	\begin{cases}
		q(x) = p(x)\dfrac{Q(x)}{P(x)} \\
		\dfrac{p'(x)}{p(x)}= \dfrac{R(x)}{P(x)}\Rightarrow \displaystyle\int \dfrac{dp}{p}=\int \dfrac{R(x)}{P(x)}dx \Rightarrow ln(p) = \int \dfrac{R(x)}{P(x)} + C \Rightarrow p(x)=e^{\int_{0}^{x}\frac{R(t)}{P(t)}dt }
	\end{cases}
\]
*$\mathcal{L}$ tự phó với tích trong
\begin{align*}
	(f,g) = \int_{a}^{b}f^{*}(x)g(x)dx
\end{align*}

Điều kiện biên
\begin{align*}
	(\mathcal{L}f,g) & =\int_{a}^{b}\left[\dfrac{d}{dx}\left( p(x)\dfrac{d\vec{f}}{dx} \right) - q(x)\vec{f}\right]g(x)dx                                                                            \\
	                 & = \left. p(x)\dfrac{df^*}{dx}g(x)\right|_a^b - \int_{a}^{b}p(x)\dfrac{df^*}{dx}\dfrac{dg}{dx} - q(x)f(x)^* g(x) dx                                                            \\
	                 & = \left. p(x)\dfrac{df^*}{dx}g(x)\right|_a^b - \left.  f^*p\dfrac{dy}{dx} \right|_a^b + \int_{a}^{b} f^*\left[ \dfrac{d}{dx}\left(p\dfrac{dg}{dx} - q(x)g(x)\right) \right]dx
\end{align*}
trong đó: $\displaystyle\int_{a}^{b} f^*\left[ \dfrac{d}{dx}\left(p\dfrac{dg}{dx} - q(x)g(x)\right) \right]dx = (f,\mathcal{L}g)$ và $ \left. p(x)\dfrac{df^*}{dx}g(x)\right|_a^b - \left.  f^*p\dfrac{dy}{dx} \right|_a^b = A$ \\
Để $(\mathcal{L}f,g) (f,\mathcal{L}g) \Rightarrow A = 0 $

\section{Tính chất của $\mathcal{L}$}
\subsection{Hàm riêng và hàm trọng}
\begin{align*}
	\mathcal{L}y(x) = \lambda w(x)y(x)
\end{align*}
hàm trọng $w(x)$ là hàm trọng và phải là thực.
\begin{align*}
	(f,g)_w \equiv \int_{a}^{b}f^*gw(x)dx
\end{align*}
tại vì $w$ là thực nên ta có tính chất:
\begin{align*}
	(f,g)_w = (f,wg) = (wf,g)
\end{align*}
Tính tuyến tính và phản tuyến tính:
\begin{align*}
	(f;c_1g_1+c_2g_2)_w = c_1(f,g_1) + c_2(f,g_2) \\
	(c_1f_1+c_2f_2;g)_w = c_1^*(f_1,g) + c_2(f_2,g)
\end{align*}
*Hàm riêng: trực giao với mỗi $\lambda$
\begin{align*}
	\int_{a}^{b}Y_m^*(x)Y_n(x)w dx = \delta_{m,n}
\end{align*}
*Hàm cơ sở trực giao
\begin{align*}
	f(x) & = \sum_{n=1}^{\infty}f_nY_n(x)         \\
	     & =\sum_{n=1}^{\infty} \int Y_m^* f(x) w \\
	     & =f_m(x) \delta_{m,n}
\end{align*}
Ex: Giải phương trình vi phân $\mathcal{L}y(x) = f(x)$
\begin{align*}
	\mathcal{L} \sum_{n}y_n Y_N & = \sum_{n}y_n\lambda_n = \sum_{n}y_n\lambda_nY_n         \\
	                            & = \sum_{n}f_n Y_n \rightarrow y_n \dfrac{f_n}{\lambda_n}
\end{align*}
\newpage
Buổi 2 : Lý thuyết Sturm–Liouville(continue)\\
$\s$Giải phương trình ma trận $Mu = \vec{f} \Rightarrow M^{-1}M u = M^{-1}\vec{f} \Rightarrow u = M^{-1}\vec{f} $\\
$\s$Toán tử của phương trình không thuần nhất:
\begin{align*}
	\mathcal{L}y(x) = f(x)
\end{align*}
$\mathcal{L}$ : Toán tử vi phân tuyến tính.\\
Ta không giản được phương trình trên như ma trận bằng cách tìm hàm riêng trị riêng.

\textbf{Cách 1: Giải phương trình vi phân thuần nhất}
\begin{align*}
	\mathcal{L}y(x) = 0.
\end{align*}
Giải phương trình trên tìm ra được $y_{com} \rightarrow y(x) = y_{com} + y_p$

\textbf{Cách 2:}
\begin{align*}
	\mathcal{L}y(x) = \f{d}{dx}\left(p(x)\f{dy}{dx}\right) - q(x)y.
\end{align*}
$\mathcal{L}$ là tự phó với tích trong $(f,g) = \dps\int f^{\s}(x) g(x) dx$
\begin{align*}
	\left(f,\mathcal{L}g\right) = \left(\mathcal{L}f,g\right).
\end{align*}
\textbf{Bước 1: Tìm hàm riêng trị riêng}
\begin{align*}
	\mathcal{L}y(x) = \lambda w(x)g(x),
\end{align*}
và đồng thời thay $\mathcal{L}y \rightarrow \f{1}{\sqrt{w(x)}} \mathcal{L}(\f{\tilde{y}}{\sqrt{w(x)}})$
\begin{align*}
	 & \mathcal{L}\left(\f{\tilde{y}}{\sqrt{w(x)}}\right) = \lambda w(x) \f{\tilde{y}}{\sqrt{w(x)}} = \sqrt{w(x)}\lambda\tilde{y} \\
	 & \Rightarrow \f{1}{\sqrt{w(x)}}\mathcal{L}\left(\f{\tilde{y}}{\sqrt{w(x)}}\right) = \lambda \tilde{y}.
\end{align*}
\textit{\textbf{Tích trong của hàm trọng}}
\begin{align*}
	\begin{cases}
		(f,g)_w & = \int f^{\s} g w(x)dx = (f,gw) = (wf,g) . \\
		(f,f)_w & = 0 \rightarrow f = 0;w \neq 0  .
	\end{cases}
\end{align*}
\textit{ \textbf{Tính chất của toán tử Sturm-Liouville}}\\
Nếu:
\begin{align*}
	\mathcal{L} y_i = \lambda w y_i(x),
	\begin{cases}
		\lambda:\;\text{thực}.                                             \\
		\left(y_i,y_j\right)_w = 0\;\text{khi}\; \lambda_i \neq \lambda_j. \\
	\end{cases}
\end{align*}
Và
\begin{align*}
	Y_i = \f{y_i}{(\sqrt{y_i y_i})_w},
\end{align*}
là tập các hàm riêng trực giao và tuyến tính $\rightarrow$ đầy đủ.

\textbf{Bước 2$\s$:}
\begin{align*}
	f(x) & = \sum_{i} f_i Y_i(x) \\
	f_i  & = (Y_i f(x))_w
\end{align*}

\textbf{Bước 2:}
\begin{align*}
	\mathcal{L} \phi(x) = w(x) F(x) \\
	\begin{cases}
		\phi(x) & = \dps\sum_{i}\phi_i Y_i(x) \\
		F(x)    & = \dps\sum_{i}F_i Y_i(x)
	\end{cases}
\end{align*}

\textbf{Bước 3:}
\begin{align*}
	\Rightarrow \mathcal{L}\sum_i \phi_i Y_i(x)                                          & = w\sum_{i}F_i Y_i(x)                     \\
	\rightarrow \sum_{i} \underbracket{\phi_i \mathcal{L} Y_i(x)}_{\lambda_i w(x)Y_i(x)} & = \sum_{i} F_i w(x) Y_i(x)                \\
	\rightarrow \sum_{i}\left( \phi_i \lambda_i - F_i \right)w(x) Y_i(x)                 & = 0 \rightarrow \phi = \f{F_i}{\lambda_i}
\end{align*}
\begin{align*}
	\phi_p (x) = \sum_{i} \f{F_i}{\lambda_i} Y_i(x) \; \text{nghiệm đặc biệt},
\end{align*}
nghiệm tổng quát cho $\phi$:
\begin{align*}
	\phi = \phi_c(c) + \phi_p(x) \; \text{đây chính là nghiệm cho phương trình} \; \mathcal{L}\phi_i = 0.
\end{align*}

\textbf{Nghiệm đặc biệt}
\begin{align*}
	\phi_p(x) & = \sum_{i} \f{1}{\lambda_i}\left( Y_i(t), F(t) \right)_w Y_i(x)                               \\
	          & = \sum_{i} \f{1}{\lambda_i} Y_i(x) \int_{a}^{b} Y^{\ast}_{i}(t) F(t)w(x)dt                    \\
	          & = \int_{a}^{b} \left[ \sum_{i} \f{1}{\lambda_i} Y^{}_{i}(x) Y^{\ast}_{i}(t) \right]w(t)F(t)dt \\
	          & = \int_{a}^{b} G(x,t)f(t)dt
\end{align*}
trong đó: $G$ là hàm Green (với 2 biến $\neq$ nhau: $x$ và $t$ và chia cho trị riêng).\\
$f(t)$: ngoại lực
$\begin{cases}
		\text{Bài toán dao động.} \\
		\text{Forcing function.}
	\end{cases}$\\
Hàm Green cho nghịch đảo hình thức.
\section{Đồng nhất thức Parseval}
\begin{align*}
	(F,F)_w & = \int \sum_{i} F^{\ast}_i Y^{}_i \sum_{j} F_j Y_j w dx                                       \\
	        & =\sum_{i,j} F^{\ast}_i F_j \underbracket{(Y_i,Y_j)_w}_{\delta_{i,j}} = \sum_{i}^{}\abs{F_i}^2
\end{align*}
\chapter{Hàm suy rộng}
\section{Hàm suy rộng và phân bố}
\textbf{Định nghĩa hàm thử}:\\
Lớp các hàm thử, chọn miền $\Omega \subseteq$ $\mathscr{R}^n$ nằm trong khôg gian có $Dim = n$. Hàm thử là các hàm trơn vô hạn $\phi = \in C^{\infty}(\Omega)$ mà tồn tại một tập compact support nằm trong ($\Omega$). Tồn tại một tập compact $K \subset \Omega$ mà làm cho $\phi(x) = 0$, khi $x \notin K$. Một ví dụ đơn giản cho hàm thử một chiều là:
\begin{align*}
	\phi(x) =
	\begin{cases}
		e^{\frac{1}{(1-x)^2}}; \quad \quad \text{khi} \; \abs{x}<1 \\
		0
	\end{cases}
\end{align*}

Gọi $\mathcal{D}(\Omega)$ là không gian các hàm thử. Sau khi đã chọn lớp các hàm thử, ta định nghĩa một phân bố $T$ là một ánh xạ tuyến tính:
\begin{align*}
	T:  \mathcal{D}(\Omega) & \rightarrow \mathscr{R} \\
	\phi                    & \rightarrow T[\phi]
\end{align*}
Tập hợp các $T\rightarrow$ không gian các phân bố $\rightarrow$ không gian hàm thử $\mathcal{D}'(\Omega)$

\chapter{Hàm Green cho phương trình đạo hàm riêng thuần nhất}
\begin{align*}
	content...
\end{align*}

\section{Bài tập trong Arfken \& Weber}


\subsubsection{10.1.1/p456}
Chứng minh rằng:
\begin{align*}
	G(x,t) =
	\begin{cases}
		 & x , \quad 0 \leq x < t, \\
		 & t , \quad t < x\leq 1 , \\
	\end{cases}
\end{align*}
là hàm Green cho toán tử $\mathcal{L} = -\dfrac{d^2}{dx^2}$, và điều kiện biên $y(0) = 0, y'(1) = 0$.\\

\textbf{Giải:}\\
Ta có $\mathcal{L} y = -\dfrac{d^2 y}{dx^2}  = -y''$. Ta đi giải phương trình thuần nhất $\mathcal{L} y = 0$.\\
Phương trình đặc trưng cho $\mathcal{L} y = 0$ là:
\begin{align*}
	y^2 = 0
\end{align*}
Nghiệm tổng quát cho $\mathcal{L} y = 0$ là:
\begin{align*}
	y_1 & = A + Bx.     \\
	y_2 & = A + B(1-x). \\
\end{align*}
Áp dụng điều kiện biên:
\begin{align*}
	 & y_1(0) = A + Bx = 0 \rightarrow A = 0 \rightarrow y_1(x) = x, \\
	 & y_{2}{'}(1) = B = 0 \rightarrow y_2(x) = 1.
\end{align*}
$Wronskian$ cho $y_1,y_2$ là:
\begin{align*}
	W = y_{1} y_{2}' - y_{2} y_{1}' = -1.
\end{align*}
Hàm Green cho $\mathcal{L} y = 0$:
\begin{align*}
	G(x,\xi)  = \Theta(\xi - x) y_1(x)y_2(\xi) + \Theta(x - \xi) y_2(x)y_1(\xi) (DPCM)
\end{align*}
\subsubsection{10.1.2}
Tìm hàm Green cho
\begin{enumerate}[label=(\alph*)]
	\item $\mathcal{L} y(x) = \dfrac{d^2 y(x)}{dx^2} + y(x) ,\quad
		      \begin{cases}
			      y(0)  & = 0, \\
			      y'(1) & = 0. \\
		      \end{cases}$
	\item  $\mathcal{L} y(x) = \dfrac{d^2 y(x)}{dx^2} - y(x) , \quad y(x)$ hữu hạn khi $-\infty < x < \infty$\\
\end{enumerate}

\textbf{Giải}\\
\begin{enumerate}[label=(\alph*)]
	\item Ta đi giải phương trình thuần nhất $\mathcal{L} y(x) = 0$. \\
	      Phương trình đặc trưng cho $\mathcal{L} y(x) = 0$ là:
	      \begin{align*}
		      y^2 + 1 = 0 ,
	      \end{align*}
	      nghiệm tổng quát cho phương trình trên là:
	      \begin{align*}
		       & y_{1} = A \cos x + B \sin x,         \\
		       & y_{2} = A \cos (1-x) + B \sin (1-x).
	      \end{align*}
	      Áp dụng điều kiện biên:
	      \begin{align*}
		       & y_{1}(0) = 0 = A \cos 0 + B \sin 0 \rightarrow A = 0 \rightarrow y_{1}(x) = \sin x ,     \\
		       & y'_{2}(1) = 0 = A \sin 0 - B \cos 0 \rightarrow B = 0 \rightarrow y_{2}(x) = \cos (1-x).
	      \end{align*}
	      $Wronskian$ cho $y_1,y_2$ là:
	      \begin{align*}
		      \begin{vmatrix}
			      y_1  & y_2  \\
			      y'_1 & y'_2 \\
		      \end{vmatrix}
		       & =
		      \begin{vmatrix}
			      \sin x & \cos (1-x) \\
			      \cos x & \sin (1-x) \\
		      \end{vmatrix}                       \\
		       & = \sin x \sin(1-x) - \cos x \cos (1-x) \\
		       & = -\cos 1 = W.
	      \end{align*}
	      Hàm Green cho phương trình $\mathcal{L} y$ với $\alpha = 1$ và $W = -\cos 1$:
	      \begin{align*}
		      G(x,\xi) = -\f{1}{\cos 1} \left[ \Theta(\xi - x) \sin x \cos(1-\xi) + \Theta(x - \xi) \cos(1-x)\sin \xi \right].
	      \end{align*}
	\item  Phương trình đặc trưng cho $\mathcal{L} y(x) = 0$:
	      \begin{align*}
		      y^2 - 1 = 0,
	      \end{align*}
	      nghiệm tổng quát cho phương trình trên là:
	      \begin{align*}
		      y_1(x) & = A e^{x} + B e^{-x},    \\
		      y_2(x) & = A e^{x-1} + B e^{1-x}.
	      \end{align*}
	      Áp dụng điều kiện biên:
	      \begin{align*}
		      y_1(x\rightarrow - \infty) & \rightarrow B = 0 \rightarrow y_1(x) = e^{x}  ,  \\
		      y_2(x\rightarrow + \infty) & \rightarrow A = 0 \rightarrow y_2(x) = e^{1-x} .
	      \end{align*}
	      $Wronskian$ cho $y_1,y_2$ là:
	      \begin{align*}
		      \begin{vmatrix}
			      y_1  & y_2  \\
			      y'_1 & y'_2 \\
		      \end{vmatrix}
		       & =
		      \begin{vmatrix}
			      e^{x} & e^{1-x}  \\
			      e^{x} & -e^{1-x} \\
		      \end{vmatrix}                     \\
		       & = - e^{x} e^{1-x} - e^{x} e^{1-x} \\
		       & = -2 e^{x} e^{1-x} = W.
	      \end{align*}
	      Hàm Green cho phương trình $\mathcal{L} y$ với $\alpha = 1$ và $W = -2 e^{x} e^{1-x}$:
	      \begin{align*}
		      G(x,\xi) 
		      &= -\f{1}{2 e^{x} e^{1-x}} \left[ \Theta(\xi - x) e^{x} e^{1-\xi} + \Theta(x - \xi) e^{\xi} e^{1-x} \right] \\
		      & = -\f{1}{2} \left[ \Theta(\xi - x) e^{x-\xi} + \Theta(x - \xi) e^{\xi - x} \right].
	      \end{align*}
\end{enumerate}
\subsubsection{10.1.4/p.457}
Tìm hàm Green cho phương trình sau:
\begin{align*}
	-\dfrac{d^2 y}{dx^2} - \dfrac{y}{4} = f(x),
\end{align*}
với điều kiện biên là $y(0) = y(\pi) = 0$. \\

\textbf{Giải:}\\
Ta đi giải phương trình thuần nhất cho $\mathcal{L} y(x) = -\dfrac{d^2 y}{dx^2} - \dfrac{y}{4}  = 0$. Phương trình đăcj trưng cho $\mathcal{L} y(x) = 0:$
\begin{align*}
	-y^2 - \dfrac{1}{4} = 0,
\end{align*}
nghiệm tổng quát cho phương trình trên là:
\begin{align*}
	y_{1}(x) &= A \cos \f{x}{2} + B \sin \f{x}{2},\\
	y_{2}(x) &= A \cos \f{x}{2} + B \sin \f{x}{2}.
\end{align*}
Áp dụng điều kiện biên:
\begin{align*}
	y_{1}(0) &= A \cos 0 + B \sin 0 \rightarrow A = 0 \rightarrow y_{1}(x) = \sin \f{x}{2}, \\
	y_{2}(\pi) &= A \cos \f{\pi}{2} + B \sin \f{\pi}{2} = 0 \rightarrow B = 0 \rightarrow y_2(x) = \cos \f{x}{2}.
\end{align*}
$Wronskian$ cho $y_1,y_2$ là:
\begin{align*}
	W = y_1 y'_2 - y_2 y'_1 
	&= -\f{1}{2} \sin \f{x}{2} \sin \f{x}{2} - \f{1}{2} \cos \f{x}{2} \cos \f{x}{2}\\
	& = -\f{1}{2} 
\end{align*}
Hàm Green cho phương trình $\mathcal{L}$ với $\alpha = - 1$ và $ W = -\f{1}{2}$:
\begin{align*}
	G(x,\xi) = 2 \left[ \Theta(\xi - x) \sin \f{x}{2} \cos \f{\xi}{2} + \Theta(x-\xi) \sin \f{\xi}{2} \cos \f{x}{2} \right]
\end{align*}

















































































































































\end{document}