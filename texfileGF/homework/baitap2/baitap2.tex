\documentclass{article}
\usepackage[utf8]{vietnam}
\usepackage[utf8]{inputenc}
\usepackage{anyfontsize,fontsize}
\changefontsize[13pt]{13pt}	
\usepackage{commath}
\usepackage{parskip}
\usepackage{xcolor}
\usepackage{amssymb}
\usepackage{slashed}
\usepackage{indentfirst}
\usepackage{pdfpages}
\usepackage{graphicx}
\usepackage{nccmath}
\usepackage{mathtools}
\usepackage{amsfonts}
\usepackage{amsmath,systeme}
\usepackage[thinc]{esdiff}
\usepackage{hyperref}
\usepackage{bm,physics}
\usepackage{fancyhdr}
%footnote
\pagestyle{fancy}
\renewcommand{\headrulewidth}{0pt}%
\fancyhf{}%
\fancyfoot[L]{Vật lý Lý thuyết}%
\fancyfoot[C]{\hspace{4cm} \thepage}%


\usepackage{geometry}
\geometry{
	a4paper,
	total={170mm,257mm},
	left=20mm,
	top=20mm,
}


\newcommand{\image}[1]{
	\begin{center}
		\includegraphics[width=0.5\textwidth]{pic/#1}
	\end{center}
}
\renewcommand{\l}{\ell}
\newcommand{\dps}{\displaystyle}

\newcommand{\f}[2]{\dfrac{#1}{#2}}
\newcommand{\at}[2]{\bigg\rvert_{#1}^{#2} }


\renewcommand{\baselinestretch}{2.0}


\title{\Huge{Bài tập về nhà 1}}

\hypersetup{
	colorlinks=true,
	linkcolor=red,
	filecolor=magenta,      
	urlcolor=cyan,
	pdftitle={GF},
	pdfpagemode=FullScreen,
}

\urlstyle{same}

\begin{document}
\setlength{\parindent}{20pt}
\newpage
\author{TRẦN KHÔI NGUYÊN \\ VẬT LÝ LÝ THUYẾT}
\maketitle
\noindent \textbf{Câu 1:}
\begin{align*}
	\f{\partial u}{\partial x} + y \f{\partial u}{\partial y} = \sin x,
\end{align*}
với điều kiện biên $u(x,0) = \cos x$. \\
\textbf{Câu 2:}
\begin{align*}
	\f{\partial u}{\partial x} - \f{\partial u}{\partial y} = u^{2},
\end{align*}
với $u = 4$, dọc theo $y = 2x - 1$.\\
\textbf{Câu 3:}
\begin{align*}
	\f{\partial u}{\partial x} + u \f{\partial u}{\partial y} = x,
\end{align*}
với điều kiện biên $u(y,0) = -2y, \forall x \in \mathbb{R}$. \\
\textbf{Bài làm}\\
\textbf{Câu 1:}
The Lagrange - Charpit equations for the PDE are:
\begin{align*}
	 & \f{dx}{1} = \f{dy}{y} = \f{du}{\sin x} \\
	 &
	\Rightarrow
	\begin{cases}
		\f{dy}{dx} = y             \\
		\f{du}{dy} = \f{\sin x}{y} \\
		\f{du}{dx} = \f{\sin x}{1}
	\end{cases}
\end{align*}
which leads to
\begin{align*}
	 & \f{dy}{y} = dx \Rightarrow \int \f{dy}{y} = \int dx \Rightarrow y = Ae^{x} \\
	 & \f{du}{dy} = \f{\sin x}{y} \Rightarrow
\end{align*}
\textbf{Câu 2:}\\
The Lagrange - Charpit equations for the PDE are:
\begin{align*}
	 & \f{dx}{1} = \f{dy}{-1} = \f{du}{u^{2}} \\
	 &
	\Rightarrow
	\begin{cases}
		\f{dy}{dx} = -1     \\
		\f{du}{dy} = -u^{2} \\
		\f{du}{dx} = u^{2}
	\end{cases},
\end{align*}
this leads to:
\begin{align*}
	dy = - dx           & \Rightarrow y = -x + A                                    \\
	-\f{du}{u^{2}} = dy & \Rightarrow \f{1}{u} = y + B \Rightarrow u = \f{1}{y + B}
\end{align*}
By applying the initial conditions.
\begin{align*}
	u(x,y) = u(x,2x - 1) = 4,
\end{align*}
we set
\begin{align*}
	\begin{cases}
		(2x - 1)\at{x = \xi}{} = - x \at{x = \xi}{} + A \\
		u = \f{1}{2x - 1 + B} \at{x = \xi}{} = 4
	\end{cases}
	\Rightarrow
	\begin{cases}
		A = 3 \xi - 1 \\
		B = -2\xi + \f{5}{4}
	\end{cases}
	\Rightarrow
	\begin{cases}
		y = - x + 3\xi - 1 \\
		u = \f{1}{y - 2\xi + \frac{5}{4}}
	\end{cases}
\end{align*}
Eliminating $\xi$ from the second equation by using the first, this yield
\begin{align*}
	u(x,y) = \f{1}{y - \frac{2}{3}\left( y + x + 1 \right) + \frac{5}{4}}
\end{align*}
is solution of the PDE.\\
\textbf{Câu 3:}
%From the Lagrange - Charpit equations for the PDE, we have:
%\begin{align*}
%	\f{dx}{1} = \f{dy}{u} = \f{du}{x}
%\end{align*}
%Further, we have
%\begin{align*}
%	\begin{cases}
%		\f{dy}{dx} = u        \\
%		\f{du}{dy} = \f{x}{u} \\
%		\f{du}{dx} = x
%	\end{cases}
%	\Rightarrow
%	\begin{cases}
%		dy = u dx   \\
%		u du = x dy \\
%		du = x dx   \\
%	\end{cases}
%	\Rightarrow
%	\begin{cases}
%		\dps\int dy =  \dps \int u dx  \\
%		\dps \int u du = x\dps \int dy \\
%		\dps \int du = \dps \int x dx
%	\end{cases}
%\end{align*}
%this leads to
%\begin{align*}
%	\begin{cases}
%		y = u x + A            \\
%		\f{u^{2}}{2} = x y + B \\
%		u = \f{x^{2}}{2} + C
%	\end{cases}
%	\Rightarrow
%	\begin{cases}
%		A = y - ux             \\
%		B = \f{u^{2}}{2} - x y \\
%		C = u - \f{x^{2}}{2}
%	\end{cases}\tag{1}
%\end{align*}
%By applying the initial conditions $u (y,0) = -2y$(replace dummpy variable $y$ by $\xi$), we set equations (1)
%\begin{align*}
%	\begin{cases}
%		A = 0 - (-2y) y \at{x = y =\xi}{}                                \\
%		B = \f{4y^{2}}{2}\at{x = y = \xi}{} - x\at{x = y = \xi}{}\times0 \\
%		C = \left(-2y - \f{x^{2}}{2}\right)\at{x=y=\xi}{}
%	\end{cases}
%	\Leftrightarrow
%	\begin{cases}
%		A = 2\xi^{2} \\
%		B = 2\xi^{2} \\
%		C = -2\xi - \frac{\xi^{2}}{2}
%	\end{cases}
%\end{align*}
%Plugging $A,B$ into (1),
%\begin{align*}
%	y            & = ux + 2\xi^{2} \tag{2}                     \\
%	\f{u^{2}}{2} & = xy + 2\xi^{2} = xy + y - ux \tag{3}       \\
%	u            & = \frac{x^2}{2} + -2\xi - \frac{\xi^{2}}{2}
%\end{align*}
%Rearranging (3), we have
%\begin{align*}
%	u^{2} + 2 u x - 2 xy - 2y = 0,
%\end{align*}
%$\Delta = $
%
%\begin{align*}
%	\f{\partial u}{\partial x} + u \f{\partial u}{\partial y} = x \Leftrightarrow \f{1}{x}\f{\partial u}{\partial x} + \f{u}{x}\f{\partial u}{\partial y} = 1
%\end{align*}
The value of function $u$ at an arbitrary $\xi$ on the initial curve ($s$ = 0) is given by $u(y,0) = -2y$. By replacing dummy variables $y \equiv \xi$, of courses, the initial condition now is $u(\xi,0) = -2\xi$.\\
At $s = 0$, the curve starts at the initial point $(x = 0 , y = \xi)$.
\begin{align*}
	\f{dy}{ds} & = u  \quad\quad \text{with} \; y(s = 0) = \xi, \tag{1}   \\
	\f{dx}{ds} & = 1  \quad\quad \text{with} \; x(s = 0) = 0, \tag{2}     \\
	\f{du}{ds} & = x  \quad\quad \text{with} \; u(s = 0) = -2\xi. \tag{3}
\end{align*}
Consider equation (2)
\begin{align*}
	dx = ds \Rightarrow x = s + A .\tag{4}
\end{align*}
Next is equation (3)
\begin{align*}
	du = x ds \Rightarrow u = (s + A) ds \Rightarrow u = \f{s^{2}}{2} + s A + C \tag{5}
\end{align*}
Plugging $u$ into (1), this leads to
\begin{align*}
	dy = u ds = \left(\f{s^{2}}{2} + s A + C\right) ds \Rightarrow y = \f{s^{3}}{6} + \f{s^{2}A}{2} + C s +  B \tag{6}
\end{align*}
Apllying initial condition to (4),(5),(6)
\begin{align*}
	\begin{cases}
		x(s = 0) = 0 = A   \\
		y(s = 0) = \xi = B \\
		u(s = 0) = -2\xi = C
	\end{cases}
	\Rightarrow
	\begin{cases}
		A = 0   \\
		B = \xi \\
		C = -2\xi
	\end{cases}
\end{align*}
We have
\begin{align*}
	x & = s                           \\
	y & = \f{s^{3}}{6} - 2\xi s + \xi =  \f{s^{3}}{6} - (2s + 1) \xi \\
	u & = \f{s^{2}}{2} - 2\xi
\end{align*}
Eliminating $\xi$, this leads to
\begin{align*}
	x & = s                                             \\
	\xi & = \f{\frac{x^{3}}{6} - y}{2s + 1}                  \\
	u & = \f{x^{2}}{2} - 2 \left( \f{\frac{x^{3}}{6} - y}{2 + 1} \right) \tag{7}
\end{align*}
Symplyfing the Eqn.(7), we have
\begin{align*}
	u(x,y) = \f{x^{2}}{2} - \f{x^{3}}{6x + 3} + \f{2y}{2x + 1}
\end{align*}
which is the solution of the PDE




\end{document}
